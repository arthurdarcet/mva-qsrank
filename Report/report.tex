\documentclass{article}
\usepackage[utf8]{inputenc}
\usepackage[english]{babel}

\textwidth 16.2cm \textheight 23cm \topmargin -0.6cm
\oddsidemargin 0.31cm \evensidemargin -0.91cm

\usepackage{natbib}
\usepackage{amsmath}
\usepackage{amsfonts}
\usepackage{amsbsy}
\usepackage{amssymb}
\usepackage{algorithm2e}
\usepackage{graphicx}
\usepackage{psfrag}
\usepackage{epsfig}
\usepackage{multicol}
\usepackage{cite}
\usepackage{color}
\usepackage{dsfont}
\usepackage[center]{caption}
\usepackage{listings} 
\usepackage{xcolor} 
\usepackage{textcomp}
\usepackage{hyperref}

\hypersetup{
    bookmarks=true,         % show bookmarks bar?
    unicode=false,          % non-Latin characters in Acrobat’s bookmarks
    pdftoolbar=true,        % show Acrobat’s toolbar?
    pdfmenubar=true,        % show Acrobat’s menu?
    pdffitwindow=false,     % window fit to page when opened
    pdfstartview={FitH},    % fits the width of the page to the window
    pdftitle={My title},    % title
    pdfauthor={Author},     % author
    pdfsubject={Subject},   % subject of the document
    pdfcreator={Creator},   % creator of the document
    pdfproducer={Producer}, % producer of the document
    pdfkeywords={keyword1} {key2} {key3}, % list of keywords
    pdfnewwindow=true,      % links in new window
    colorlinks=true,       % false: boxed links; true: colored links
    linkcolor=red,          % color of internal links (change box color with linkbordercolor)
    citecolor=blue,        % color of links to bibliography
    filecolor=magenta,      % color of file links
    urlcolor=cyan,           % color of external links
	pdfborder	= {0 0 0}
}

\def\endproof{\hfill $\Box$\newline\newline}
\def\proof{\par\noindent{\it Proof}. \ignorespaces}

\newcommand{\RR}{\mathbb{R}}
\newcommand{\qr}{\textrm{QsRank}}

\title{Qs Rank and efficient $\epsilon$-neighbor search}
\author{Arthur Darcet \& Yohann Salaun}
\date{\today}

\parindent=0pt
\begin{document}
\maketitle

\section{Overview}

The QsRank algorithm, described in \citep{QSRank}, is a method that allows a ranking for binary hash codes to efficiently perform $\epsilon$-neighbors search in large data. The aim of this problem is to, for a given query $q$ and a given radius $\epsilon$, quickly finds the data subset $Y = (y_i)_i$ such that $\forall i,\  ||y_i-q|| < \epsilon$.\\
Methods based on the Hamming distance have been recently developed to solve efficiently such a problem for large data. However these latter methods often lack of precision due to their binarization of the query which induces a lack of speed in the performances of their algorithm. The methods presented here, performs a more accurate ranking for hash codes and thus allows a faster and more efficient method for $\epsilon$-neighbors retrieval.

\section{Theoretical Description}

The algorithm of $\epsilon$-neighbor search using the QsRank works in two times:
\begin{itemize}
	\item[\textbf{1.}] Hash code generation of the data for faster retrieval
	\item[\textbf{2.}] Ranking of the hash code depending on a query
\end{itemize}
The first part can be computed before any requests whereas the second needs the parameters of the search (query $q$ and neighbor distance $\epsilon$).

\subsection{Hash Codes generation}

Such search algorithm are used for huge data with components in high dimensional space. Thus, in order to generate simple hash codes for each input data point $x \in \RR^d$, a Principal Component Analysis (PCA) is computed.\\
The hash code $h$ of $x$ is then computed by taking the sign of the PCA-projections:
\[	
\forall j \in [1;d],\ h_j = 	
 	\left \{
		\begin{array}{c}
    		1 \text{ if }(PCA(x, \{x_i\}_i))_j > 0 \\
    		0 \text{ if }(PCA(x, \{x_i\}_i))_j \leq 0
		\end{array}
	\right.
\]
In practice, only the first components of the hash codes are computed.\\
The benefits given by the PCA are numerous:
\begin{itemize}
	\item[\textbf{1.}] The PCA allows a dimension decrease but keeps most of the information. This way, hash codes are shorter and the retrieval is still efficient.
	\item[\textbf{2.}] The PCA is an orthogonal projection that preserves the $L^2$-norm. Thus, the $\epsilon$-ball around a query is still meaningful after PCA.
	\item[\textbf{3.}] The PCA values of the input data points are uncorrelated which will lead to an efficient ranking with Qs Rank.
\end{itemize}

\subsection{QsRank for Hash Codes ranking}

We suppose, that the projection $y$ of the input data points $x$ are distributed along a probability distribution function $p$. Then, with a given query $q$, a neighbor distance $\epsilon$ and a hash code $h$, the Qs Rank formula is defined by:
\[
	\qr(q, h, \epsilon) = \frac{\int_{NN(q,\epsilon) \cap S(h)} p(y) dy}{\int_{NN(q,\epsilon)} p(y) dy}
\]
where:
\begin{enumerate}
	\item[$\bullet$]$NN(q,\epsilon) = \{y \in \RR^d \text{ s.t. } ||y-q||<\epsilon\}$ is the $\epsilon$-ball around the query $q$
	\item[$\bullet$]$S(h) = \{y \in \RR^d \text{ s.t. } \forall i \in [1;d] y_i h_i > 0 \}$ is the set described by the hash code $h$
\end{enumerate}

The QsRank can only be seen as a probability, with Bayes rule:
\[
	\qr(q, h, \epsilon) = \frac{\mathbb{P}(y\in NN(q,\epsilon) \cap S(h))}{\mathbb{P}(y\in NN(q,\epsilon))} = \mathbb{P}(y \in S(h) | y\in NN(q,\epsilon))
\]
Thus, the QsRank only ranks hash codes with respect to their probability of containing many $\epsilon$-neighbors.

\subsection{QSRank approximation}

In order to compute fast retrieval, the QsRank will be approximated by a lighter formula.\\
First, only the top $k$ dimensions of the PCA projection will be used ($k$ will be defined afterward in the next section). Thus, $NN(q,\epsilon)$ becomes $NN(q^k,\epsilon)$ and $S(h)$ becomes $S(h^k)$ where $x^k$ is the k-top dimensions of a vector $x \in \RR^d$. This approximation seems legit since the aim of the PCA is to find the dimensions where most of the information is kept.
\[
	\qr(q, h, \epsilon) \approx \frac{\int_{NN(q^k,\epsilon) \cap S(h^k)} p(y^k) dy^k}{\int_{NN(q^k,\epsilon)} p(y^k) dy^k}
\]
Another approximation is to replace the $\epsilon$-ball by an $\epsilon$-hypercube:
\[
	NN(q^k,\epsilon) \leftrightarrow HC(q^k,\epsilon) = \{y^k \in \RR^k \text{ s.t. } \forall i \in [1;k], |y^k_i-q^k_i|<\epsilon\}
\]
Moreover, since the PCA produces uncorrelated projections, each dimension of $p(y)$ is supposed to be independent. The QsRank approximation then becomes:
\[
	\qr(q, h, \epsilon) \approx \prod_{i=1}^k \frac{\int_{|y^k_i - q^k_i| < \epsilon, y^k_i h^k_i > 0 } p(y^k_i) dy^k_i}{\int_{|y^k_i - q^k_i| < \epsilon} p(y^k_i) dy^k_i} = \prod_{i=1}^k \mathbb{P}(y^k_i \in S(h^k_i) | y^k_i \in HC(q^k_i,\epsilon))
\]
The last approximation is to consider that the $y$ are generated from a uniform law. This assumption accelerates a lot the computation and seems to work quite well in accordance with the authors of \citep{QSRank}. The final formula thus becomes:
\begin{eqnarray*}
	\qr(q, h, \epsilon) 
	& \approx & \prod_{i=1}^k \frac{\int_{|y^k_i - q^k_i| < \epsilon, y^k_i h^k_i > 0 } dy^k_i}{\int_{|y^k_i - q^k_i| < \epsilon} dy^k_i} \\
	& \approx & \prod_{i=1}^k \text{clamp} \left(\frac{1}{2}\left( 1 + \frac{h^k_i q^k_i}{\epsilon} \right), [0;1] \right)                  
\end{eqnarray*}

Opposed to the Hamming distance, this measure has many substantial advantages :
\begin{enumerate}
	\item[$\bullet$] The radius $\epsilon$ is took into account	
	\item[$\bullet$] Since it is a product, if one of the component of the hash code induces a null probability, the whole QsRank becomes null. Some sets are thus not explored whereas Hamming distance methods would have.
\end{enumerate}

\section{Implementation}

\subsection{Efficient computation}

Once the query $q$ is given, the logarithmic QsRank is computed for the $2^k$ different hash codes. This results in $\mathcal{O}(k)$ logarithm computation and $\mathcal{O}(k 2^k)$ additions. Moreover, only hash codes with non zeros probabilities have to be computed, which accounts for only $15\%$ of the data points according to the experiments made in \citep{QSRank}.

\subsection{Retrieval procedure}

\begin{algorithm}[H]

\textbf{Input} : input data $X \in \RR^{n \times d}$, query $q$, radius $\epsilon$\\
\textbf{Parameters}: hash codes length $K_1$ and $K_2$, number of buckets $L$.\\
\textit{Precomputation}\\
$Y = PCA(X)$\\
\For{$i=1~..~K_1+K_2$}{
	\textbf{Compute} the logarithmic sub-QsRanks : $\log\left(\text{clamp} \left(\frac{1}{2}\left( 1 + \frac{h^k_i q^k_i}{\epsilon} \right), [0;1] \right)\right)$
}
\textit{$\epsilon$-search}\\
\textbf{Sort} the buckets by $K_1$-QsRank.\\
\textbf{Pick} the $L^{th}$ first buckets.\\
\textbf{Sort} the buckets hash codes by $(K_1+K_2)$-QsRank.\\
\textbf{Return} the $\epsilon$-neighbors.
\end{algorithm}


\section{Results}

\bibliographystyle{plain} %Style of Bibliography: plain / apalike / amsalpha / ...
\bibliography{literature} %You need a file 'literature.bib' for this.

\end{document}
