\documentclass{article}
\usepackage[utf8]{inputenc}
\usepackage[english]{babel}

\textwidth 16.2cm \textheight 23cm \topmargin -0.6cm
\oddsidemargin 0.31cm \evensidemargin -0.91cm

\usepackage{natbib}
\usepackage{amsmath}
\usepackage{amsfonts}
\usepackage{amsbsy}
\usepackage{amssymb}
\usepackage{algorithm2e}
\usepackage{graphicx}
\usepackage{psfrag}
\usepackage{epsfig}
\usepackage{multicol}
\usepackage{cite}
\usepackage{color}
\usepackage{dsfont}
\usepackage[center]{caption}
\usepackage{listings} 
\usepackage{xcolor} 
\usepackage{textcomp}
\usepackage{hyperref}

\hypersetup{
    bookmarks=true,         % show bookmarks bar?
    unicode=false,          % non-Latin characters in Acrobat’s bookmarks
    pdftoolbar=true,        % show Acrobat’s toolbar?
    pdfmenubar=true,        % show Acrobat’s menu?
    pdffitwindow=false,     % window fit to page when opened
    pdfstartview={FitH},    % fits the width of the page to the window
    pdftitle={My title},    % title
    pdfauthor={Author},     % author
    pdfsubject={Subject},   % subject of the document
    pdfcreator={Creator},   % creator of the document
    pdfproducer={Producer}, % producer of the document
    pdfkeywords={keyword1} {key2} {key3}, % list of keywords
    pdfnewwindow=true,      % links in new window
    colorlinks=true,       % false: boxed links; true: colored links
    linkcolor=red,          % color of internal links (change box color with linkbordercolor)
    citecolor=blue,        % color of links to bibliography
    filecolor=magenta,      % color of file links
    urlcolor=cyan,           % color of external links
	pdfborder	= {0 0 0}
}

\def\endproof{\hfill $\Box$\newline\newline}
\def\proof{\par\noindent{\it Proof}. \ignorespaces}

\title{Qs Rank and efficient $\epsilon$-neighbor search}
\author{Arthur Darcet \& Yohann Salaun}
\date{\today}

\parindent=0pt
\begin{document}
\maketitle

\section{Overview}

Article about \citep{QSRank}.

\section{Theoretical Description}

The algorithm of $\epsilon$-neighbor search using the Qs Rank works in two times:
\begin{itemize}
	\item[\textbf{1.}] Hash code generation of the data for faster retrieval
	\item[\textbf{2.}] Ranking of the hash code depending on a query
\end{itemize}
The first part can be computed before any requests whereas the second needs the parameters of the search (query $q$ and neighbor distance $\epsilon$).

\subsection{Hash Codes generation}

Such search algorithm are used for huge data with components in high dimensional space. Thus, in order to generate simple hash codes for each input data point $x \in \mathbb{R}^d$, a Principal Component Analysis (PCA) is computed.\\
Each Hash Code $h$ is then computed by taking the sign of the PCA-projections:
\[	
\forall j \in [1;d],\ h_j = 	
 	\left \{
		\begin{array}{c}
    		1 \text{ if }(PCA(x))_j > 0 \\
    		0 \text{ if }(PCA(x))_j \leq 0
		\end{array}
	\right.
\]
In practice, only the first components of the hash codes are computed.\\
The benefits given by the PCA are numerous:
\begin{itemize}
	\item[\textbf{1.}] The PCA allows a dimension decrease but keeps most of the information. This way, hash codes are shorter and the retrieval is still efficient.
	\item[\textbf{2.}] The PCA is an orthogonal projection that preserves the $L^2$-norm. Thus, the $\epsilon$-ball around a query is still meaningful after PCA.
	\item[\textbf{3.}] The PCA values of the input data points are uncorrelated which will lead to an efficient ranking with Qs Rank.
\end{itemize}

\subsection{QsRank for Hash Codes ranking}

\subsection{QSRank approximation}

\section{Implementation}

\subsection{Efficient computation}

\subsection{Retrieval procedure}

Pseudo code ?

\section{Results}


\bibliographystyle{plain} %Style of Bibliography: plain / apalike / amsalpha / ...
\bibliography{literature} %You need a file 'literature.bib' for this.

\end{document}
